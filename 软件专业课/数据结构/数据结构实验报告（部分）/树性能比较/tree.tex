\documentclass[]{article}
\usepackage{ctex}
\usepackage{setspace}
\usepackage{graphicx}
%opening
\title{BST,AVL,BT,RBT性能对比}
\author{}

\begin{document}

\maketitle


\begin{large}
	\heiti
	一、 结构说明:
\end{large}
\zihao{5}

1. 一般的二叉搜索树(BST): 

二叉搜索树是一种基于二叉树的数据结构,其中每个节点最多有两个子节点,分别称为左子节点和右子节
点。对于每个节点,其左子树上的所有节点的键值小于该节点的键值,而右子树上的所有节点的键值大于
该节点的键值,这是二叉搜索树的关键性质。 

2. AVL 树: 

AVL 树是一种自平衡的二叉搜索树,它在每个节点上维护一个平衡因子,以确保树的左子树和右子树的高
度差不超过1。平衡性质使得 AVL 树具有较快的查找性能,因为树的高度保持较小。AVL 树通过旋转操作
来保持平衡,包括左旋、右旋、左右旋和右左旋。比较严格的平衡要求可能导致频繁的旋转操作,影响性
能。 

3. B 树: 

B 树是一种多路搜索树,每个节点可以包含多个键值和子节点。B 树的节点被设计为可以容纳更多的键值,
以适应大量的数据。B 树广泛应用于文件系统和数据库系统中,以提供高效的范围查询和高度平衡的树结
构。 

4. 红黑树: 

红黑树是一种自平衡的二叉搜索树,具有额外的红黑色属性。每个节点都有一个颜色(红色或黑色)并满
足一组红黑规则,包括根节点和叶子节点(通常是 NIL 节点)是黑色的,相邻节点不能同时为红色等。红
黑树的平衡性质保证了在树的高度范围内,插入、查找和删除操作的性能。实现相对简单,维护性较好,

\begin{large}
	\heiti
	二,算法核心
\end{large}
\zihao{5}

1 按递增顺序插入N个整数,随机查找1000个数,按相同顺序删除 

2 按递增顺序插入N个整数,随机查找1000个数,按相反顺序删除 

3 按随机顺序插入N个整数,随机查找1000个数,按随机顺序删除 

N从10000到400000取值,间隔为20000,对比不同树的操作性能 

\begin{large}
	\heiti
	三,数据结果 
\end{large}

\begin{figure}[h]
	
	\centering
	\includegraphics[width=\linewidth]{"C:/Users/the dark/Pictures/Screenshots/屏幕截图 2024-11-24 164944.png"}
	\caption{乱序插入性能表}
	\label{fig:enter-label}

		\centering
	\includegraphics[width=\linewidth]{"C:/Users/the dark/Pictures/Screenshots/屏幕截图 2024-11-24 165431.png"}
	\caption{顺序插入性能表}
	\label{fig:enter-label}

	\centering
\includegraphics[width=\linewidth]{"C:/Users/the dark/Pictures/Screenshots/屏幕截图 2024-11-24 222711.png"}
\caption{查找性能表}
\label{fig:enter-label}
\end{figure}
\begin{figure}[h]
	\centering
\includegraphics[width=\linewidth]{"C:/Users/the dark/Pictures/Screenshots/屏幕截图 2024-11-24 165711.png"}
\caption{随机删除性能表}
\label{fig:enter-label}

\end{figure}
\begin{figure}[h]
	\centering
\includegraphics[width=\linewidth]{"C:/Users/the dark/Pictures/Screenshots/屏幕截图 2024-11-26 222909.png"}
\caption{顺序删除性能表}
\label{fig:enter-label}

\end{figure}
\begin{figure}[h]
	\centering
\includegraphics[width=\linewidth]{"C:/Users/the dark/Pictures/Screenshots/屏幕截图 2024-11-25 000644.png"}
\caption{逆序删除性能表}
\label{fig:enter-label}
\end{figure}

\zihao{5}

对实验报告最后图表的说明:
注意到三种插入删除方式有些共通点,故将有共通点的数据进行了合并,如1和2方式的插入均是顺序插入,故将顺序插入的数据放在一张表中取平均值进行制图,查找同理,但查找中BST的随机插入查找和顺序插入查找差距太大,故将BST分为两种情况讨论,BST1是随机插入的随机寻找,BST2是三种插入方式的查找取平均值

由于BST的顺序插入及逆序删除的时间太久,BST的顺序插入查找时间太久,BT的顺序删除的时间太久,故涉及以上数据的表均给出两张图,分别是带以上数据和不带以上数据的图

由于测试的数据由随机数生成器mt19937生成,范围为1到2n(n为数据量),故测试数据中含有重复元素,但几种树使用的是同一组数据,所以不存在哪种树的数据的重复元素更多的情况,但不同数据量组之间的数据会有一定差异,考虑到程序预热,内存分配等情况下面的讨论均对图的趋势做讨论,忽略图中的一些极端数据(表现为图中的某个点突出一个尖)

\begin{large}
	\heiti
	四,分析总结
\end{large}
\zihao{5}

1, 在顺序递增插入中,由于 BST 插入形成一颗斜着的树,每次插入都是 O(n)
的时间复杂度,所以所需时间非常非常高久,而AVL需要大量的旋转,所以
耗费时间比RBT更久,而BT插入需要合并节点,拷贝节点,耗时较久

2, 在随机查找中,若是递增插入,则BST是一颗斜着的树,所耗时间太久。随机插入情况下BST,AVL,RBT三者难舍难分,主要是因为它们都是二叉搜索树,时间复杂度均为logn级,而39w组的数据或许还是太少,没能让AVL控制树高的优势发挥出来
对于B树,其查找大多是在数组内连续单元访问,比其他树用指针访问的速度更快一些。 

3, 顺序插入后,正向删除,BST因为是一棵斜着的树,每次检索都是检索到的第一个元素,所耗时间是最短的,
之后是RBT和AVL,原因与插入一样,
RBT 的旋转调整次数更少,而BT由于删除过于复杂,树的结构不断改变,
所以是最久的 

4, 顺序插入后逆向删除,BST是最久的,因为每次都要检索到最后一个元素再
删除。其他三个树与顺序删除是一样的。 

5, 随机插入中,首先
是RBT,旋转次数很少,所以比BT和AVL都快,而AVL因为需要旋转次数
太多了,所以比BT需要的更久,而且是随机插入所以AVL需要旋转次数比
顺序插入更多。 而BST虽然无需任何调整,但其树高不平衡,会比RBT和AVL的平均树高更高,所以插入的总体表现与RBT相似


6, AVL最慢,因为随机删除它
需要多次旋转。最快的仍然是RBT,因为查找本身比BST快
不少,而删除并不需要很多的旋转,所耗时间最短。 而BT虽然树高低,查找速度快,但它的合并节点等操作依然复杂,故与RBT和BST难舍难分

综上所述:根据实验数据,BST在递增插入、递增插入后随机查找,递增插入
后递减删除四种情况下性能表现较差,BT在随机插入、随机插入后随机查找、随
机插入后随机删除四种情况下表现较差,也有可能是数据未在硬盘读取的问题,而
AVL 与RBT在各种情况下均有良好的性能表现。

\begin{large}
	\heiti
	五.原始数据
\end{large}
\begin{figure}[h]
	\centering
	\includegraphics[width=\linewidth]{"C:/Users/the dark/Pictures/Screenshots/屏幕截图 2024-11-26 223325.png"}
	\caption{原始数据}
	\label{fig:enter-label}
\end{figure}
\zihao{5}

每种树的1为随机插入,随机删除,2为递增插入,顺序删除,3为递增插入,逆序删除,每组数据的第一行是插入,第二行是查找,第三行是删除
\end{document}
